\chapter{Tests}

The Physicalc test suite consists of three parts: 1)~interactive
programs used during development; 2)~unit tests implemented with
JUnit; and 3)~integration tests that test the whole interpreter,
implemented with Bash shell scripts.

\section{Interactive Testing During Development}

Several executable Java programs assisted in developing and debugging
the interpreter.  These files remain in the source tree as
\texttt{src/Try*.java}.

\textbf{TryLexer} runs the Physicalc lexer on a string, given as a
command line argument, and prints out the tokens, one per line.

\textbf{TryParser} runs the lexer and parser on a string, given as a
command line argument, and prints out the abstract syntax tree in a
Lisp-like syntax.

\textbf{ParseFile} acts like TryParser, but takes a file name as an
argument and parses the file.

\textbf{TryDatum} executes various arithmetical operations on the
Datum sub-classes and prints the results.

TryDatum was written by Brian Foo; TryLexer, TryParser, and ParseFile
were written by Stuart Sierra.



\section{Unit Tests}

To test the operation of simple expressions in the interpreter, tests
defined using JUnit\cite{junit} are defined in
\texttt{test/InterpreterTest.java}.  This class defines an
``assertPrints'' method which calls the interpreter on a string of
source code and checks that it produces a certain output.  Due to
Java's lack of support for multi-line strings in source code, this
mode of testing is awkward for longer programs.  The Makefile target
\texttt{test} compiles the interpreter and runs the unit tests.  The
unit tests framework was written by Stuart Sierra and tests were added
by Changlong Jiang.



\section{Integration Tests}

Tests of the complete behavior of the interpreter are implemented with
the \texttt{runexamples} shell script.  This script looks for files
named \texttt{*.in} and \texttt{*.out} in the directory
\texttt{test/examples}.  The ``in'' files are Physicalc source code
containing one or more ``print'' statements.  The corresponding
``out'' files contain the text that the program should print.  The
\texttt{runexamples} script runs the interpreter on each ``in'' file
and compares its output with its ``out'' file, reporting how they
differ.  The \texttt{runexamples} script was written by Stuart Sierra;
test programs were contributed by all team members.  The example files
are included in the source code section at the end of this chapter.
