\begin{thebibliography}{99}

\bibitem{calchemy} Calchemy, \url{http://www.calchemy.com/}

\bibitem{si} ``International System of Units.''  Wikipedia,
  \url{http://en.wikipedia.org/wiki/Si_units}

\bibitem{rounding} ``Rounding numbers.''  Wikipedia,
  \url{http://en.wikipedia.org/wiki/Rounding_numbers}

\bibitem{base-unit} ``SI base unit.''  Wikipedia,
  \url{http://en.wikipedia.org/wiki/SI_base_unit}

\bibitem{derived-unit} ``SI derived unit.''  Wikipedia,
  \url{http://en.wikipedia.org/wiki/SI_derived_unit}

\bibitem{units} Units, GNU Project,
  \url{http://www.gnu.org/software/units/units.html}

\bibitem{apfloat} Apfloat, \url{http://www.apfloat.org/apfloat_java/}

\bibitem{octave} GNU Octave, \url{http://www.gnu.org/software/octave/}

\bibitem{gcalc} Google Calculator, \url{http://www.google.com/help/calculator.html}

\bibitem{jas} Java Algebra System, \url{http://krum.rz.uni-mannheim.de/jas/}

\bibitem{javadoc} How to Write Doc Comments for the Javadoc Tool, Sun
  Developer Network,
  \url{http://java.sun.com/j2se/javadoc/writingdoccomments}

\bibitem{javastyle} Code Conventions for the Java Programming
Language, Sun Developer Network, \url{http://java.sun.com/docs/codeconv/}

\bibitem{jscience} JScience,  \url{http://jscience.org/}

\bibitem{jscl} Jscl-meditor, \url{http://jscl-meditor.sourceforge.net/}

\bibitem{junit} JUnit, \url{http://www.junit.org/}

\bibitem{matlab} MATLAB, \url{http://www.mathworks.com/products/matlab/}

\bibitem{maxima} Maxima, \url{http://maxima.sourceforge.net/}

\bibitem{sunmass} Physics Problem: Calculating the Mass of the Sun.
  University of Oregon,
  \url{http://zebu.uoregon.edu/~probs/mech/grav/Gravity2/Gravity2.html}

\bibitem{moonorbit} Physics Problem: Calculating the Radius of the Moon's Orbit.
  University of Oregon,
  \url{http://zebu.uoregon.edu/~probs/mech/grav/distmoon/distmoon.html}


\end{thebibliography}
