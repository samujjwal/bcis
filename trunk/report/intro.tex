\chapter{Introduction}

Physicalc is a programming language for scientific computation,
designed for students studying beginning and intermediate-level
physics, chemistry, or other sciences.  

Computer algebra systems are typically oriented towards higher
mathematics, making them ill-suited to the sorts of calculations done
by high-school and undergraduate science students.  At the same time,
some basic computer algebra features, such as symbolic computation,
could be helpful to students.  Physicalc presents itself initially as
an intelligent calculator that understands physical units like
``meters/second.''  For more advanced users, it supports real
programming in an imperative style.

Physicalc is intended primarily as an educational tool, but may also
be useful for exploratory data analysis in scientific fields.

\section{Language Overview}

\subsection{Interpreter}

Physicalc is an interpreted programming language. The interpreter is
written in Java and can be run either interactively or on a file
containing Physicalc source code.  The interface is text-mode,
although a GUI could be layered on top of it.

\subsection{Syntax}

Physicalc syntax is as simple as possible, using mostly English words,
more reminiscent of BASIC than C\@.  Statements are separated by
newlines.  Statement blocks are enclosed in ``do\dots done'' pairs.
Standard imperative-language features such as loops, if/then/else
branching, and user-defined functions are provided. Standard
mathematical operators are provided, with the addition of `\verb+^+'
for exponentiation.

\section{Prior Art}

\begin{itemize}

\item Units\cite{units}, a command-line program included in early
\acro{unix} systems and \acro{gnu}/Linux, provides conversion factors
between various units.

\item Calchemy\cite{calchemy} is software for the Palm OS that combines a
scientific calculator with unit conversion and dimensional analysis.
It is based on an earlier Windows program called \acro{unicalc}.

\item The Google Calculator\cite{gcalc} performs arithmetic on numbers
including unit conversions.

\item JScience\cite{jscience} is a Java library for scientific computation
including units.

\end{itemize}
